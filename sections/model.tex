\newpage

\section{Model}
\label{sec:Model}

Let's consider a magnetic system which consists of two subsystems and the transition region.

The first subsystem is consists of the ferromagnetic films separated by the nonmagnetic layers. The width of the separators determines the intensity and the type of the interaction between magnetic films. We will consider the antiferromagnetic type of the interaction which means that the vectors of magnetization of the neighboring magnetic layers are aligned oppositely. This subsystem we will call the polarizer.

The structure of the second subsystem is identical to the first one. We will call it the analyzer. The main difference between the polarizer and the analyzer is that the magnetization of the former is fixed by some external reason and the magnetization of the latter is free.

The two subsystems are arranged one after the other in the plane of magnetic films, so that against each magnetic layer of analyzer is  (напротив каждого магнитного слоя анализатора находиться феромагнитный слой поляризатораб намагниченностя которых направлена противоположно)

\begin{figure}[h]
	\centering
	\newpage

Let's consider a magnetic system which consists of two subsystems and the transition region.

The first subsystem is consists of the ferromagnetic films separated by the nonmagnetic layers. The width of the separators determines the intensity and the type of the interaction between magnetic films. We will consider the antiferromagnetic type of the interaction which means that the vectors of magnetization of the neighboring magnetic layers are aligned oppositely. This subsystem we will call the polarizer.

The structure of the second subsystem is identical to the first one. We will call it the analyzer. The main difference between the polarizer and the analyzer is that the magnetization of the former is fixed by some external reason and the magnetization of the latter is free.

The two subsystems are arranged one after the other in the plane of magnetic films, so that against each magnetic layer of analyzer is  (напротив каждого магнитного слоя анализатора находиться феромагнитный слой поляризатораб намагниченностя которых направлена противоположно)

\begin{figure}[h]
	\centering
	\newpage

Let's consider a magnetic system which consists of two subsystems and the transition region.

The first subsystem is consists of the ferromagnetic films separated by the nonmagnetic layers. The width of the separators determines the intensity and the type of the interaction between magnetic films. We will consider the antiferromagnetic type of the interaction which means that the vectors of magnetization of the neighboring magnetic layers are aligned oppositely. This subsystem we will call the polarizer.

The structure of the second subsystem is identical to the first one. We will call it the analyzer. The main difference between the polarizer and the analyzer is that the magnetization of the former is fixed by some external reason and the magnetization of the latter is free.

The two subsystems are arranged one after the other in the plane of magnetic films, so that against each magnetic layer of analyzer is  (напротив каждого магнитного слоя анализатора находиться феромагнитный слой поляризатораб намагниченностя которых направлена противоположно)

\begin{figure}[h]
	\centering
	\newpage

Let's consider a magnetic system which consists of two subsystems and the transition region.

The first subsystem is consists of the ferromagnetic films separated by the nonmagnetic layers. The width of the separators determines the intensity and the type of the interaction between magnetic films. We will consider the antiferromagnetic type of the interaction which means that the vectors of magnetization of the neighboring magnetic layers are aligned oppositely. This subsystem we will call the polarizer.

The structure of the second subsystem is identical to the first one. We will call it the analyzer. The main difference between the polarizer and the analyzer is that the magnetization of the former is fixed by some external reason and the magnetization of the latter is free.

The two subsystems are arranged one after the other in the plane of magnetic films, so that against each magnetic layer of analyzer is  (напротив каждого магнитного слоя анализатора находиться феромагнитный слой поляризатораб намагниченностя которых направлена противоположно)

\begin{figure}[h]
	\centering
	\input{figures/model.pdf_tex}
	\caption{Schema of the model}
	\label{fig:model}
\end{figure}

These two subsystems are separated by the transition region, the magnetization of which is relatively small. This region we will call the interface. The width of the interface is determined by two factors: it should be sufficiently large to prevent direct exchange interaction between subsystems; also it should be sufficiently small to provide electrons transitions between subsystems in the ballistic mode.

The electron current density flowing through the polarizer along z axis gains the spin-polarization which is aligned parallel to the magnetization of the appropriate magnetic layer, so the polarization is changed periodically along x axis due to periodicity of the structure. The appropriate influence of the current to the magnetic structure is neglected due to the magnetization fixing which was mentioned earlier. Thus, we obtain spin-polarized current which flows into the analyzer through the interface.

Flowing through the analyzer the electro









	\caption{Schema of the model}
	\label{fig:model}
\end{figure}

These two subsystems are separated by the transition region, the magnetization of which is relatively small. This region we will call the interface. The width of the interface is determined by two factors: it should be sufficiently large to prevent direct exchange interaction between subsystems; also it should be sufficiently small to provide electrons transitions between subsystems in the ballistic mode.

The electron current density flowing through the polarizer along z axis gains the spin-polarization which is aligned parallel to the magnetization of the appropriate magnetic layer, so the polarization is changed periodically along x axis due to periodicity of the structure. The appropriate influence of the current to the magnetic structure is neglected due to the magnetization fixing which was mentioned earlier. Thus, we obtain spin-polarized current which flows into the analyzer through the interface.

Flowing through the analyzer the electro









	\caption{Schema of the model}
	\label{fig:model}
\end{figure}

These two subsystems are separated by the transition region, the magnetization of which is relatively small. This region we will call the interface. The width of the interface is determined by two factors: it should be sufficiently large to prevent direct exchange interaction between subsystems; also it should be sufficiently small to provide electrons transitions between subsystems in the ballistic mode.

The electron current density flowing through the polarizer along z axis gains the spin-polarization which is aligned parallel to the magnetization of the appropriate magnetic layer, so the polarization is changed periodically along x axis due to periodicity of the structure. The appropriate influence of the current to the magnetic structure is neglected due to the magnetization fixing which was mentioned earlier. Thus, we obtain spin-polarized current which flows into the analyzer through the interface.

Flowing through the analyzer the electro









	\caption{Schema of the model}
	\label{fig:model}
\end{figure}

These two subsystems are separated by the transition region, the magnetization of which is relatively small. This region we will call the interface. The width of the interface is determined by two factors: it should be sufficiently large to prevent direct exchange interaction between subsystems; also it should be sufficiently small to provide electrons transitions between subsystems in the ballistic mode.

The electron current density flowing through the polarizer along z axis gains the spin-polarization which is aligned parallel to the magnetization of the appropriate magnetic layer, so the polarization is changed periodically along x axis due to periodicity of the structure. The appropriate influence of the current to the magnetic structure is neglected due to the magnetization fixing which was mentioned earlier. Thus, we obtain spin-polarized current which flows into the analyzer through the interface.

Further, we will consider the influence of this current on the magnetic states of the analyzer.

Flowing through the bulk of the analyzer the moving electrons scatter on the magnetic sublattice. This can be effectively described as a torque acting on the analyzers magnetization.

For the sake of the simple analytical describing of the analyzer let's divide it into two sublattices. Each sublattice contains all magnetic layers with the same direction of the magnetization. Here we assume that only two opposite directions is allowed in equilibrium state.

So, we can consider each sublattice as a simple ferromagnetic system, and apply standard approach to depict the influence of spin-polarized current on it.
\be
	\label{eq:LLGS}
	\dot{\mb{M}}_k =
			- \gamma \left[ \mb{M}_k \times \mb{H}_k \right]
			+ \frac{\alpha_G}{M_0}
				\left[
					\mb{M}_k \times \dot{\mb{M}}_k
				\right]
			+ \frac{\sigma I}{M_0}
				\left[
					\mb{M}_k \times \left[ \mb{M}_k \times \uv{p}_k \right]
				\right].
\ee
Here $k$ denotes sublattice. In our case $k = 1, 2$. $\mb{M}_k$ is a vector of magnetization which is defined as a magnetic moment of the unit volume of the appropriate sublattice. $\uv{p}$ is a current polarization vector -- unit vector, directed along the polarization of the current in sufficient sublattice. $\mb{H}_k = -\partial W / \partial \mb{M}_k$ is an effective magnetic field acting on $k$-th sublattice, $W$ is a density of the free energy of the analyzer. It is important to mention that it contains the term which describe the interaction between sublattices.

As a first touch, we consider isotropic ferromagnetic material. Therefor, the density of the free energy can be considered as following:
\be
	\label{eq:free_enegy}
	W = J_\text{int} \bra{ \mb{M}_1 \cdot \mb{M}_2 }
		- \frac{1}{2} K \bra{ M_{1z}^2 + M_{2z}^2 }.
\ee
This quantity achieves it minimum value when the magnetizations of the sublattices directs oppositely to each other and along the $z$ axis. The former statement says that we deal with antiferromagnetic interaction between sublattices, and the latter statement defines preferred direction in the structure of the ferromagnetic material. According to this assumption $p_k = (0, 0, (-1)^{k+1})$.

