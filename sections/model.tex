\newpage

Let's consider a magnetic system which consists of two subsystems and the transition region.

The first subsystem is consists of the ferromagnetic films separated by the nonmagnetic layers. The width of the separators determines the intensity and the type of the interaction between magnetic films. We will consider the antiferromagnetic type of the interaction which means that the vectors of magnetization of the neighboring magnetic layers are aligned oppositely. This subsystem we will call the polarizer.

The structure of the second subsystem is identical to the first one. We will call it the analyzer. The main difference between the polarizer and the analyzer is that the magnetization of the former is fixed by some external reason and the magnetization of the latter is free.

The two subsystems are arranged one after the other in the plane of magnetic films, so that against each magnetic layer of analyzer is  (напротив каждого магнитного слоя анализатора находиться феромагнитный слой поляризатораб намагниченностя которых направлена противоположно)

\begin{figure}[h]
	\centering
	\newpage

Let's consider a magnetic system which consists of two subsystems and the transition region.

The first subsystem is consists of the ferromagnetic films separated by the nonmagnetic layers. The width of the separators determines the intensity and the type of the interaction between magnetic films. We will consider the antiferromagnetic type of the interaction which means that the vectors of magnetization of the neighboring magnetic layers are aligned oppositely. This subsystem we will call the polarizer.

The structure of the second subsystem is identical to the first one. We will call it the analyzer. The main difference between the polarizer and the analyzer is that the magnetization of the former is fixed by some external reason and the magnetization of the latter is free.

The two subsystems are arranged one after the other in the plane of magnetic films, so that against each magnetic layer of analyzer is  (напротив каждого магнитного слоя анализатора находиться феромагнитный слой поляризатораб намагниченностя которых направлена противоположно)

\begin{figure}[h]
	\centering
	\newpage

Let's consider a magnetic system which consists of two subsystems and the transition region.

The first subsystem is consists of the ferromagnetic films separated by the nonmagnetic layers. The width of the separators determines the intensity and the type of the interaction between magnetic films. We will consider the antiferromagnetic type of the interaction which means that the vectors of magnetization of the neighboring magnetic layers are aligned oppositely. This subsystem we will call the polarizer.

The structure of the second subsystem is identical to the first one. We will call it the analyzer. The main difference between the polarizer and the analyzer is that the magnetization of the former is fixed by some external reason and the magnetization of the latter is free.

The two subsystems are arranged one after the other in the plane of magnetic films, so that against each magnetic layer of analyzer is  (напротив каждого магнитного слоя анализатора находиться феромагнитный слой поляризатораб намагниченностя которых направлена противоположно)

\begin{figure}[h]
	\centering
	\newpage

Let's consider a magnetic system which consists of two subsystems and the transition region.

The first subsystem is consists of the ferromagnetic films separated by the nonmagnetic layers. The width of the separators determines the intensity and the type of the interaction between magnetic films. We will consider the antiferromagnetic type of the interaction which means that the vectors of magnetization of the neighboring magnetic layers are aligned oppositely. This subsystem we will call the polarizer.

The structure of the second subsystem is identical to the first one. We will call it the analyzer. The main difference between the polarizer and the analyzer is that the magnetization of the former is fixed by some external reason and the magnetization of the latter is free.

The two subsystems are arranged one after the other in the plane of magnetic films, so that against each magnetic layer of analyzer is  (напротив каждого магнитного слоя анализатора находиться феромагнитный слой поляризатораб намагниченностя которых направлена противоположно)

\begin{figure}[h]
	\centering
	\input{figures/model.pdf_tex}
	\caption{Schema of the model}
	\label{fig:model}
\end{figure}

These two subsystems are separated by the transition region, the magnetization of which is relatively small. This region we will call the interface. The width of the interface is determined by two factors: it should be sufficiently large to prevent direct exchange interaction between subsystems; also it should be sufficiently small to provide electrons transitions between subsystems in the ballistic mode.

The electron current density flowing through the polarizer along z axis gains the spin-polarization which is aligned parallel to the magnetization of the appropriate magnetic layer, so the polarization is changed periodically along x axis due to periodicity of the structure. The appropriate influence of the current to the magnetic structure is neglected due to the magnetization fixing which was mentioned earlier. Thus, we obtain spin-polarized current which flows into the analyzer through the interface.

Flowing through the analyzer the electro









	\caption{Schema of the model}
	\label{fig:model}
\end{figure}

These two subsystems are separated by the transition region, the magnetization of which is relatively small. This region we will call the interface. The width of the interface is determined by two factors: it should be sufficiently large to prevent direct exchange interaction between subsystems; also it should be sufficiently small to provide electrons transitions between subsystems in the ballistic mode.

The electron current density flowing through the polarizer along z axis gains the spin-polarization which is aligned parallel to the magnetization of the appropriate magnetic layer, so the polarization is changed periodically along x axis due to periodicity of the structure. The appropriate influence of the current to the magnetic structure is neglected due to the magnetization fixing which was mentioned earlier. Thus, we obtain spin-polarized current which flows into the analyzer through the interface.

Flowing through the analyzer the electro









	\caption{Schema of the model}
	\label{fig:model}
\end{figure}

These two subsystems are separated by the transition region, the magnetization of which is relatively small. This region we will call the interface. The width of the interface is determined by two factors: it should be sufficiently large to prevent direct exchange interaction between subsystems; also it should be sufficiently small to provide electrons transitions between subsystems in the ballistic mode.

The electron current density flowing through the polarizer along z axis gains the spin-polarization which is aligned parallel to the magnetization of the appropriate magnetic layer, so the polarization is changed periodically along x axis due to periodicity of the structure. The appropriate influence of the current to the magnetic structure is neglected due to the magnetization fixing which was mentioned earlier. Thus, we obtain spin-polarized current which flows into the analyzer through the interface.

Flowing through the analyzer the electro









	\caption{Schema of the model}
	\label{fig:model}
\end{figure}

These two subsystems are separated by the transition region, the magnetization of which is relatively small. This region we will call the interface. The width of the interface is determined by two factors: it should be sufficiently large to prevent direct exchange interaction between subsystems; also it should be sufficiently small to provide electrons transitions between subsystems in the ballistic mode.

The electron current density flowing through the polarizer along z axis gains the spin-polarization which is aligned parallel to the magnetization of the appropriate magnetic layer, so the polarization is changed periodically along x axis due to periodicity of the structure. The appropriate influence of the current to the magnetic structure is neglected due to the magnetization fixing which was mentioned earlier. Thus, we obtain spin-polarized current which flows into the analyzer through the interface.

Flowing through the analyzer the electro








