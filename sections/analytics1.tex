\newpage

\section{Analytics}
\label{sec:Analytics}

\subsection{Linear approximation}
According to the model which was described in the section \ref{sec:Model} the equilibrium states of the magnetization vector in each sublattice directed along $z$ axis:
$$
	\mb{M}^0_k = \bra{0, 0, (-1)^{k+1}M_0}
$$
where $M_0$ -- saturation value of the magnetization.

Let's assume that the magnetization vector is slightly deviated from the equilibrium states by the value which is much smaller then $M_0$. This excited vector can be written as a sum of two perpendicular vectors:
$$
	\mb{M}_k = \mb{m}_k + \mb{M}_{kz},
$$
where $\mb{m}_k$ is a small vector which lies on $xy$ plane.

The equations of motion (\ref{eq:LLGS}) leave the magnitude of the magnetization vector constant, so $\abs{\mb{M}_k} = M_0$. In this case, the magnitude of the $\mb{M}_{kz}$ can be approximated as following:
\bea
	\mb{M}_{kz}			&=& \mb{M}_k - \mb{m}_k; 						\nn \\
	M_{kz}^2 			&=& M_k^2 - 2\mb{M}_k \cdot \mb{m}_k + m_k^2 	\nn \\
						&=& M_k^2 - m_k^2 = M_0^2 - m_k^2; 				\nn \\
	\abs{\mb{M}_{kz}} 	&=& \sqrt{M_0^2 - m_k^2} = M_0 \sqrt{1-\frac{m_k^2}{M_0^2}} \nn
\eea

The expending of the last expression in the Taylor series gives	$M_{kz} \approx M_0 \bra{1-\frac{1}{2}\frac{m_i^2}{M_0^2}}$. Hence, in linear approximation the magnetization vector can be considered as following:
\be
\label{eq:magn}
	\mb{M}_k = \bra{m_{kx}, m_{kx}, (-1)^{k+1}M_0}
\ee

In our case the current flowing through the bulk is a source of the external amount of energy which manifesting in the deviations of the magnetization vector from it equilibrium state. The internal damping dissipate this energy. So, there is a value of the current when this two processes compensate each other. This value is called critical current. Let's find it analytical expression.

By substituting the expression for magnetization vector (\ref{eq:magn}) into the equations of motion (\ref{eq:LLGS}) and neglecting the terms of the \emph{second order of smallness} (Here we assume that the small quantity is one of the vectors $\mb{m}_k$ or it component) we obtain a system of equations for deviation terms:
\bea
	\dot{m}_{1x} &=&	- \sigma I							m_{1x}
						- \gamma M_0 ( J_\text{int} + K )	m_{1y}
						- \gamma M_0 J_\text{int}			m_{2y}
						- \alpha_G							\dot{m}_{1y}	\nn \\
	\dot{m}_{1y} &=&	  \gamma M_0 ( J_\text{int} + K )	m_{1x}
						- \sigma I							m_{1y}
						+ \gamma M_0 J_\text{int}			m_{2x}
						+ \alpha_G							\dot{m}_{1x}	\nn \\
	\dot{m}_{2x} &=&	  \gamma M_0 J_\text{int}			m_{1y}
						- \sigma I							m_{2x}
						+ \gamma M_0 ( J_\text{int} + K )	m_{2y}
						+ \alpha_G							\dot{m}_{2y}	\nn \\
	\dot{m}_{2y} &=&	- \gamma M_0 J_\text{int}			m_{1x}
						- \gamma M_0 ( J_\text{int} + K )	m_{2x}
						- \sigma I							m_{2y}
						- \alpha_G							\dot{m}_{2x}	\nn
\eea

The solutions of this system of equations we will seek in the exponent form $m_i \sim e^{i\omega t}$. In this case the eigenvalues of the frequency is following:
\bea
	\label{eq:analytic:eigenfrequency}
	\omega &=&	i \frac{ \sigma I + \gamma M_0 (K + J_\text{int}) \alpha_G }{\alpha_G^2 + 1} \\
				&\pm& \frac{\sqrt{ (\gamma M_0 (K + J_\text{int}) - \alpha_G \sigma I)^2 - (\gamma M_0 J_\text{int})^2( \alpha_G^2 + 1 )} }{ \alpha_G^2 + 1 } \nn
\eea

This expression is complex value.
The real part of it is responsible for the frequency of precession of magnetization vector around the preferred direction in the ferromagnetic layer. 
The imaginary part is responsible for either increasing or decreasing of the amplitude of precession. It is determined by the ratio between amount of pumping and damping energy.

Let's consider the situation when there is no friction in the system ($\alpha_G = 0$). In this case the eigenfrequencies of such system is determined by the expression:
$$
	\omega_\text{nofric} = i \sigma I \pm
						   \gamma M_0 \sqrt{ K \bra{ K + 2J_\text{int} } }.
$$
The real part of this frequency increase with either increasing of antiferromagnetic interaction between the layers ( $J_\text{int}$ ) or increasing of \emph{preferred direction quality}.

The imaginary part depends only on current intensity. Depending on the direction of the current flow it can play role of the effective damping or the effective pumping (it determined by the sign of $I$).

In the general case the situation is more complicated. According to the expression $\ref{eq:analytic:eigenfrequency}$ nonzero friction leads to the dependency of the real part of frequency both on the coefficient of damping and the current intensity. The pure frictional terms tends to reduce the frequency, however the term which contains current intensity can either enhance or reduce the frequency depending on it sign.

As for imaginary part, here situation is similar. The damping tends to turn the magnetization in the equilibrium position and minimize energy. The influence of current can be either dumping style (work in the same direction with damping, and as a result increasing effective damping in the system) or pumping style, when it work in opposite direction with damping, hence it work as a source of energy, so tends to increase deviations of magnetization from it equilibrium position.

When the current is working in pumping style it is possible curious situation when the current compensates energy losses in the system from the friction. In this case, the intensity of the current is called critical current. This possible when the imaginary part in the (\ref{eq:analytic:eigenfrequency}) is nullified. Hence the critical current is determined by the following expression:
\be
\label{eq:analytic:critical_current}
	I_\text{cr} = - \alpha_G \frac{\gamma M_0 \bra{ K + J_\text{int}}}{\sigma}.
\ee
As we can see, in our notation this value is negative.

It is important to mention that the anisotropy and antiferromagnetic constants determine the scale of energy in the system. According to the expression for critical current, the damping effects is proportional to sum of these two, so it proportional to the amount of energy in the system. It is also important to mention that critical current is inversely proportional to the $\sigma$. This parameter is influenced by the range of polarization of the current, the ability of the system to perceive the spin current, and it independent from the $K$ and $J_\text{int}$.













